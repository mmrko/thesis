% !TEX root = thesis.tex

\documentclass[thesis.tex]{subfiles}

\begin{document}

\addcontentsline{toc}{chapter}{Abbreviations and Acronyms}
\chapter*{Abbreviations and Acronyms}

\noindent
\begin{longtable}{@{}p{0.25\textwidth}p{0.7\textwidth}@{}}
API & Application Programming Interface \\
APS-C & Advanced Photo System type-C; image sensor format \\
CCD & Charged Coupling Devices; image sensor technology \\
CFA & Color Filter Array \\
CIE & International Commission on Illumination \\
CLI & Command Line Interface \\
CMOS & Complementary Metal Oxide Semiconductor; image sensor technology \\
CSS & Cascading Style Sheets; a style sheet language used for describing the visual look of structured markup \\
DBMS & a software package designed to define, manipulate, retrieve and manage data in one or several databases \\
DNG & Digital Negative; a lossless RAW image format by Adobe designed to improve RAW image portability \\
DSLR & Digital Single-Lens Reflex Camera \\
DSC & Digital Still Camera \\
HTML & HyperText Markup Language; a standard markup language for webpages \\
IDE & Integrated Development Environment \\
IEC & International Electrotechnical Commission \\
JS & JavaScript; a dynamic programming language commonly used for authoring client-side web applications \\
LDR & Light-Dependent Resistor; a resistor whose resistance decreases with increasing incident light intensity \\
MBaaS & Mobile Backend as a service; a cloud computing service model \\
MP & A megapixel; million pixels \\
NFC & Near-Field communication; a form of short-range wireless communication \\
OIS & Optical Image Stabilization \\
OS & Operating System \\
QR & Quick Response Code; a two-dimensional barcode \\
RGB & Red, Green, Blue color model \\
SaaS & Software as a service; a cloud computing service model \\
SDK & Software Development Kit \\
UI & User Interface \\
WK & WebKit; an open-source web browser engine used by Apple's Safari and older versions of Google's Chrome\\

\end{longtable}

\end{document}