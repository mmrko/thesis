% !TEX root = ../thesis.tex

% Tässä osassa esitetään tulokset ja vastataan tutkielman alussa
% esitettyihin tutkimuskysymyksiin. Tieteellisen kirjoitelman
% arvo mitataan tässä osassa esitettyjen tulosten perusteella.

% You have done your work, but that’s1 not enough.
% You also need to evaluate how well your implementation works. The
% nature of the evaluation depends on your problem, your method, and your
% implementation that are all described in the thesis before this chapter. If
% you have created a program for exact-text matching, then you measure how
% long it takes for your implementation to search for different patterns, and
% compare it against the implementation that was used before. If you have
% designed a process for managing software projects, you perhaps interview
% people working with a waterfall-style management process, have them adapt
% your management process, and interview them again after they have worked
% with your process for some time. See what’s changed.
% The important thing is that you can evaluate your success somehow.
% Remember that you do not have to succeed in making something spectacular;
% a total implementation failure may still give grounds for a very good master’s
% thesis—if you can analyze what went wrong and what should have been done.

\documentclass[thesis.tex]{subfiles}

\begin{document}

\chapter{Results}
\label{chapter:results}





Run similarity matrix through affinity propagation (R's APCluster)

The development of LumiNova resulted from the demands placed upon Nemoto  Co. Ltd.,  by the Japanese watch and clock manufacturers which did not want to use radioactive luminous dials in their products
Bhattacharyya distance
- binary thresholding (threshold value 40)
- (peak vallye delta 0.2)
Samsung S4 (4.4.2, API level 19)
Lumia 1020
Yongnuo YN565EX (5600K temp)
MFD
SP <-> O (similar)
dimensionality reduction

Various possible configurations (permutations), vast amount of data. Thus, top down approach. Broad level results -> some peculiar cases

We want clusters that
- dont include too many/few taggants (preferably 3-6)
- are equal in value (y-axis)
- are of the same/similar color

\ref{equation:fingerprint-penalty}
comparison vs. histogram approach
similarity parameters (damping...)

\end{document}