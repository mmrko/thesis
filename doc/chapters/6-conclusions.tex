% !TEX root = ../thesis.tex

\documentclass[thesis.tex]{subfiles}

\begin{document}

\chapter{Conclusions}
\label{chapter:conclusions}

This thesis explored the application of photoluminescence for production authentication purposes in a mobile context. A colorimetrical and temporal analysis of photoluminescence was performed for constructing unique fingerprints that could be easily utilized as a product authentication mechanism. An external camera module was built to normalize differences between various mobile devices and to provide a reproducible testing environment for capturing the photoluminescence. Two different algorithms were used for the fingerprints analysis: a custom-implemented fingerprint method based on the Hungarian algorithm and an open-source histogram-based method. To facilitate the requirements of end-to-end product authentication needs a proof-of-concept cloud architecture was implemented. This included storage and linkage of fingerprints and products as well data synchronization to the mobile device for offline support. The application was implemented as a mobile hybrid application for Android and the Windows Phone.

As the industry lacks cost-effective solutions, the integration of smartphones and affordable luminophores for production authentication purposes is of interest. The findings of this thesis indicate, however, that the technology is not yet ready for large scale adoption. Outstanding challenges concern both the current smartphone technology and the photoluminescent material (luminophores) used as the base for the analysis. As smartphones (cameras) differ in their interpretation of color (RGB) and the respective APIs provide varying levels of capture functionality, the fingerprint matching is better performed on a per-model basis. This however has implications for scalability. The luminophores on the other hand need to have good resistance against external effects (e.g., photobleaching) and properties suitable for product authentication purposes (e.g., narrow absorption wavelength area, fast decay time and integrable with other techniques). The combination of such properties requires artificial fabrication of luminophores, but also negatively affects the cost-effectiveness of the solution as the fabrication process can be tedious and costly.

Despite the remaining challenges the solution developed in this thesis remains applicable for small-scale, internal authentication needs, e.g., as an additional authentication measure. As smartphone camera APIs continue to mature better burst capture capabilities and RAW support are likely to become standard features. The storage and maintenance of per-model fingerprints can be addressed by streamlining the fingerprint creation process and by employing continuous integration practices, the latter of which is a distinct advantage of smartphone-based solutions over techniques that require specialized readers. Perhaps in the future luminophores such as tunable fluorescent materials will help in tweaking luminophores for different needs cost-effectively. Furthermore, emerging mobile technologies such as modular smartphones like Google Ara will make it possible to seamlessly integrate the camera module into the phone further lowering the barrier of entry for photoluminescence-based product authentication. For now however, NFC looks to be the most promising technology for smartphone based production authentication. As a standardized technology it is becoming a default feature in today's smartphone, which helps adoption as no external accessory is required. It remains to be seen, however, which area of the product authentication landscape its inherent, overt nature and the relatively high unit cost of secure and durable NFC tags can cater for. Perhaps a future smartphone-based production authentication solution will find a way to combine the best features of both NFC and \emph{LuminoTrace}.

\end{document}
