% !TEX root = ../thesis.tex

\documentclass[thesis.tex]{subfiles}

% Johdanto selvittaa samat asiat kuin tiivistelma, mutta laveammin. Johdannossa kerrotaan yleensa seuraavat asiat

\begin{document}

\chapter{Introduction}
\label{chapter:intro}

The past decade has seen the rapid growth of e-commerce fueled by the likes of Amazon, eBay Etsy and Alibaba. In quest of pushing products to consumers at an increasing pace and convenience, services like \emph{1-Click Purchase} and \emph{Same-Day Delivery} are becoming ever more common. Even the more traditional retailer giants like Walmart and IKEA have hopped on the e-commerce bandwagon in the hopes of increasing global sales volumes and to defend their marketshare. In the near future, the supply chain between the retailer and the consumer promises to become ever more streamlined as automation (drones and droids) takes over Last Mile deliveries.

The influx of products available for purchase both online and offline has also been seen in the steady increase of counterfeit goods on the market. Estimations by the International Chamber of Commerce (ICC) back in 2008 indicated that the value of internationally traded counterfeit products would reach upto 1770 billion US dollars by 2015 \cite{icc}. The adversarial effects of counterfeiting and piracy do not only affect the brand owner but the consumer as well. While, at worst fake, flawed items can be outright dangerous, they also ultimately lead to measurable tax income losses and other societal disbenefits. Counterfeiting and piracy concern multiple stakeholders from suppliers and brand owners to consumers and authorities.

The number of counterfeit goods on the market can be reduced by introducing a variety of preventive product authentication measures into the product lifecycle. Overt security features such as 1D/2D barcodes, watermarks and signatures/serial codes are straightforward and widely adopted techniques for \emph{identifying} products uniquely. However, they are also easily compromised. Thus, for improved security covert techniques like microprinting, biometrics and ultra-violet/infrared inks or other materials invisible to the bare eye should be employed. The tradeoff is that covert authentication techniques often require an external, specialized reader and are thus less cost-effective. Ideally, the authentication measures taken employ multiple techniques and additional metadata (e.g., track-and-trace information).

Traditionally counterfeiters have focused on high-end products but low-cost, high-volume commodities are becoming a popular target as well. While brand owners might be willing to tradeoff low-cost and ease of use for greater security (confidentiality and robustness) to protect their high-end products, for high-volume items the cost-effectiveness and simplicity of the product authentication solution are often of great importance. Ultimately the choice of the product authentication technology to employ is a compromise between usability, reliability and cost.

\enlargethispage{2\baselineskip}
The evolution of counterfeit techniques calls for new anti-counterfeit solutions suitable for various use-cases. This thesis introduces \emph{LuminoTrace} -- a photoluminescence-based product authentication technology for smartphones -- that seeks to combine a covert, low-cost and portable feature set for high-security product authentication needs. The goal is to develop a portable method (a smartphone application) for analysing photoluminescent material, or \emph{luminophores}, embedded onto the coating of a material, covertly. Multiple luminophores are combined to create unique \emph{taggants}. A colorimetrical and temporal analysis of a taggants' emission (photoluminescene) is performed to construct a \emph{fingerprint} that can be used to uniquely authenticate a product.

Figure \ref{figure:production_authentication_landscape} illustrates how \emph{LuminoTrace} seeks to fit into the cost-efficient

\begin{figure}[h!]
\centering \includegraphics[width=\linewidth]{images/pa_landscape}
\vspace{-8mm}
\caption{LuminoTrace seeks to fit into the cost-efficient end of today's product authentication landscape.}
\label{figure:production_authentication_landscape}
\end{figure}

\noindent end of today's product authentication landscape. The ideal target groups include low to medium priced, high-volume products that can sacrifice the highest level of security for significantly reduced operational costs. As an additional differentiator to some of the existing product authentication solutions the application is implemented as an \emph{offline-first} hybrid mobile application.

This thesis was commissioned by Aalto CHEM as part of their MerAito project (\emph{Merkintäteknologiset menetelmät aitouden ja jäljitettävyyden osoittamiseksi}) in the Tekes\footnote{Tekes: \url{https://www.tekes.fi/en/tekes/}} Corporate Security 2007-2013 program. The overarching goal of the project was to investigate ways to apply photoluminescene for product authentication and tagging purposes -- on a mobile device in particular -- based on the related concepts studied earlier in \cite{kuosmanen}.

The main research questions of the thesis are summarized briefly in Chapter \ref{chapter:research-questions}. The rest of this thesis is organized as follows: Chapter \ref{chapter:background} begins with an introduction to the related concepts, which include photoluminescence, color models, an overview of mobile camera technology and the hybrid mobile ecosystem. Chapter \ref{chapter:design-implementation} continues with a discussion of the design and implementation of the product authentication solution developed. Chapters \ref{chapter:experiment} and \ref{chapter:discussion} present the experiment setup, results and findings as well as the challenges faced. Last, Chapter \ref{chapter:conclusions} concludes with a summary of the topics discussed and an outline for the future.

The focus of this thesis is on \emph{smartphone based} product authentication solutions. For a more exhaustive overview of related product authentication technologies the reader is referred to \cite{kuosmanen}.

\section{Research Questions}
\label{chapter:research-questions}

This thesis aims to answer the following questions:

\begin{itemize}
  \item \label{RQ1} \textbf{RQ1}: How can photoluminescence and smartphones be integrated for product authentication purposes?
  \item \label{RQ2} \textbf{RQ2}: Which factors affect the analysis of photoluminescent material (luminophores) in the context of smartphones?
  \item \label{RQ3} \textbf{RQ3}: Does modern smartphone technology provide the means for photoluminescence based product authentication in practice?
\end{itemize}


\end{document}