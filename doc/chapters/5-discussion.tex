% !TEX root = ../thesis.tex

\documentclass[thesis.tex]{subfiles}

\begin{document}

\chapter{Discussion}
\label{chapter:discussion}

- Algorithmic Robustness
- Performance Analysis
- Problem Space and Challenges
https://publications.theseus.fi/bitstream/handle/10024/90986/Bing%20Dai-thesis.pdf

% Tutkimustuloksien merkitystä on aina syytä arvioida ja tarkastella
% kriittisesti. Tässä osassa on syytä myäs arvioida tutkimustulosten luotettavuutta.

% At this point, you will have some insightful thoughts on your implementation
% and you may have ideas on what could be done in the future. This chapter
% is a good place to discuss your thesis as a whole and to show your professor
% that you have really understood some non-trivial aspects of the methods you
% used. . .

% https://www.cs.unc.edu/~welch/media/pdf/Ilie2005_Calib.pdf
% Unfortunately most cameras—even of the same type—
% do not exhibit consistent responses. Figure 1 illustrates the
% differences between the responses of 8 cameras to the 24
% colors of the GretagMacbeth [5] ColorCheckerTM chart
% imaged under the same illumination conditions and using
% the same hardware settings. The data shows that color
% values are significantly different from camera to camera.
% This is due for example to aperture variations, fabrication
% variations, electrical noise, and interpolation artifacts arising
% from the reconstruction of a full-resolution color image
% from a half-resolution Bayer pattern image

- alternative approach: colored dots form a discrete pattern, no grating necessary
- camera parameter configuration to ensure scalability

- a fast background-independent retrieval strategy for color image databases

- more efficient matching if done server-side (leverage histogram data directly?)
No offical API, no fine grained control over the camera (ISO, shutter speed)
JavaCPP, Xamarin
Developer convenience
Future work (isot kokonaisuudet!)
NFC
- availability of NFC-enabled smartphones (minor)
- unit cost of tags

\end{document}