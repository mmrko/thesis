\documentclass[12pt,a4paper,oneside,pdftex]{report}
\usepackage[utf8]{inputenc}
\usepackage[OT1]{fontenc}
\usepackage[finnish,swedish,english]{babel}
\usepackage[square,sort&compress,numbers]{natbib}
\usepackage{eurosym}
\usepackage{verbatim}
\usepackage{longtable}
\usepackage{subfigure}
\usepackage{amssymb}
\usepackage[medium]{titlesec}

% Custom
\usepackage{subfiles}
\usepackage[bottom]{footmisc}
\newcommand{\mytilde}{\raise.17ex\hbox{$\scriptstyle\mathtt{\sim}$}}

% The aalto-thesis package provides typesetting instructions for the
% standard master's thesis parts (abstracts, front page, and so on)
% Load this package second-to-last, just before the hyperref package.
% Options that you can use:
%   mydraft - renders the thesis in draft mode.
%             Do not use for the final version.
%   doublenumbering - [optional] number the first pages of the thesis
%                     with roman numerals (i, ii, iii, ...); and start
%                     arabic numbering (1, 2, 3, ...) only on the
%                     first page of the first chapter
%   twoinstructors  - changes the title of instructors to plural form
%   twosupervisors  - changes the title of supervisors to plural form
\usepackage[mydraft,doublenumbering,twoinstructors]{aalto-thesis}

\RequirePackage[pdftex]{hyperref}
\hypersetup{colorlinks=false,raiselinks=false,breaklinks=true}
\hypersetup{pdfborder={0 0 0}}
\hypersetup{bookmarksnumbered=true}
\hypersetup{bookmarksopen=true,bookmarksopenlevel=1}

% ------------------------------------------------------------------
\newcommand{\TITLE}{LuminoTrace: photoluminescence based product authentication for smartphones}
\newcommand{\FTITLE}{LuminoTrace: fotoluminesenssiin perustuva aitouden tunnistaminen älypuhelimissa}

\newcommand{\SUBTITLE}{}
\newcommand{\FSUBTITLE}{}

\newcommand{\DATE}{March 21, 2016}
\newcommand{\FDATE}{21. maaliskuuta 2016}


% Supervisors and instructors
% ------------------------------------------------------------------
\newcommand{\SUPERVISOR}{Professor Petri Vuorimaa}
\newcommand{\FSUPERVISOR}{Professori Petri Vuorimaa}

\newcommand{\INSTRUCTOR}{Jari Kleimola D.Sc. (Tech.), Aalto SCI\\
Jouni Paltakari Professor (Tech.), Aalto CHEM}
\newcommand{\FINSTRUCTOR}{Jari Kleimola TkT, Aalto SCI\\
Jouni Paltakari Professori (tek.), Aalto CHEM}

% Other stuff
% ------------------------------------------------------------------
\newcommand{\PROFESSORSHIP}{Media Technology}
\newcommand{\FPROFESSORSHIP}{Viestintätekniikka}

% Professorship code is the same in all languages
\newcommand{\PROFCODE}{AS-75}

\newcommand{\KEYWORDS}{photoluminescence, smartphone, mobile application, authenticity}
\newcommand{\FKEYWORDS}{fotoluminesenssi, älypuhelin, mobiiliapplikaatio, aitouden tunnistaminen}

\newcommand{\LANGUAGE}{English}
\newcommand{\FLANGUAGE}{Englanti}

% Author is the same for all languages
\newcommand{\AUTHOR}{Marko Raatikka}

% Currently the English versions are used for the PDF file metadata
% Set the PDF title
\hypersetup{pdftitle={\TITLE\ \SUBTITLE}}
% Set the PDF author
\hypersetup{pdfauthor={\AUTHOR}}
% Set the PDF keywords
\hypersetup{pdfkeywords={\KEYWORDS}}
% Set the PDF subject
\hypersetup{pdfsubject={Master's Thesis}}

% \linespread{1} % This is the default
\linespread{1}

% Bibliography style
\bibliographystyle{acm}
% Plainnat is a plain style that works with both numbered and name citations.
% \bibliographystyle{plainnat}


% Extra hyphenation settings
% ------------------------------------------------------------------
% You can list here all the files that are not hyphenated correctly.
% You can provide many \hyphenation commands and/or separate each word
% with a space inside a single command. Put hyphens in the places where
% a word can be hyphenated.
% Note that (by default) LaTeX will not hyphenate words that already
% have a hyphen in them (for example, if you write ``structure-modification
% operation'', the word structure-modification will never be hyphenated).
% You need a special package to hyphenate those words.
\hyphenation{di-gi-taa-li-sta yksi-suun-tai-sta}



% The preamble ends here, and the document begins.
% Place all formatting commands and such before this line.
% ------------------------------------------------------------------
\begin{document}
% This command adds a PDF bookmark to the cover page. You may leave
% it out if you don't like it...
\pdfbookmark[0]{Cover page}{bookmark.0.cover}
% This command is defined in aalto-thesis.sty. It controls the page
% numbering based on whether the doublenumbering option is specified
\startcoverpage

% Cover page
% ------------------------------------------------------------------
% Options: finnish, english, and swedish
% These control in which language the cover-page information is shown
\coverpage{english}


% Abstracts
% ------------------------------------------------------------------
% Include an abstract in the language that the thesis is written in,
% and if your native language is Finnish or Swedish, one in that language.

% Abstract in English
% ------------------------------------------------------------------
\thesisabstract{english}{}

\thesisabstract{finnish}{}

\selectlanguage{english}

\addcontentsline{toc}{chapter}{Acknowledgments}

\chapter*{Acknowledgments}

\vskip 10mm

\noindent Espoo, \DATE
\vskip 5mm
\noindent\AUTHOR

\cleardoublepage

% Acronyms
% ------------------------------------------------------------------

\subfile{abbreviations.tex}

% Table of contents
% ------------------------------------------------------------------
\cleardoublepage
% This command adds a PDF bookmark that links to the contents.
% You can use \addcontentsline{} as well, but that also adds contents
% entry to the table of contents, which is kind of redundant.
% The text ``Contents'' is shown in the PDF bookmark.
\pdfbookmark[0]{Contents}{bookmark.0.contents}

\begingroup
\makeatletter
% Redefine the \chapter* header macro to remove vertical space
\def\@makeschapterhead#1{%
  %\vspace*{50\p@}% Remove the vertical space
  {\parindent \z@ \raggedright
    \normalfont
    \interlinepenalty\@M
    \Huge \bfseries  #1\par\nobreak
    \vskip 40\p@
  }}
\makeatother

\tableofcontents
\endgroup

% List of tables
% ------------------------------------------------------------------
% You only need a list of tables for your thesis if you have very
% many tables. If you do, uncomment the following two lines.
% \cleardoublepage
% \listoftables

% Table of figures
% ------------------------------------------------------------------
% You only need a list of figures for your thesis if you have very
% many figures. If you do, uncomment the following two lines.
% \cleardoublepage
% \listoffigures

% The following label is used for counting the prelude pages
\label{pages-prelude}
\cleardoublepage

%%%%%%%%%%%%%%%%% The main content starts here %%%%%%%%%%%%%%%%%%%%%
% ------------------------------------------------------------------
% This command is defined in aalto-thesis.sty. It controls the page
% numbering based on whether the doublenumbering option is specified
\startfirstchapter

% Add headings to pages (the chapter title is shown)
\pagestyle{headings}

% The contents of the thesis are separated to their own files.
% Edit the content in these files, rename them as necessary.
% ------------------------------------------------------------------
\subfile{chapters/1-intro.tex}

\subfile{chapters/2-background.tex}

\subfile{chapters/3-design-impl.tex}

\subfile{chapters/4-evaluation.tex}

\subfile{chapters/5-discussion.tex}

\subfile{chapters/6-conclusions.tex}

% Load the bibliographic references
% ------------------------------------------------------------------
\bibliography{references}


% Appendices go here
% ------------------------------------------------------------------
% If you do not have appendices, comment out the following lines
\appendix
\subfile{appendices.tex}

% End of document!
% ------------------------------------------------------------------
% The LastPage package automatically places a label on the last page.
% That works better than placing a label here manually, because the
% label might not go to the actual last page, if LaTeX needs to place
% floats (that is, figures, tables, and such) to the end of the
% document.
\end{document}
