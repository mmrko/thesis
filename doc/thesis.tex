\documentclass[12pt,a4paper,oneside,pdftex]{report}
\usepackage[utf8]{inputenc}
\usepackage[OT1]{fontenc}
\usepackage[finnish,swedish,english]{babel}
\usepackage[square,sort&compress,numbers]{natbib}
\usepackage{eurosym}
\usepackage{verbatim}
\usepackage{longtable}
\usepackage{subfigure}
\usepackage{amssymb}
\usepackage{seqsplit}
\usepackage{enumitem}
\usepackage{listings}
\usepackage[normalem]{ulem}
\usepackage[medium]{titlesec}

\usepackage{xcolor}

\colorlet{punct}{red!60!black}
\definecolor{background}{HTML}{EEEEEE}
\definecolor{delim}{RGB}{20,105,176}
\colorlet{numb}{magenta!60!black}

\lstdefinelanguage{json}{
    basicstyle=\footnotesize\ttfamily,
    numbers=left,
    numberstyle=\footnotesize,
    stepnumber=1,
    numbersep=9pt,
    showstringspaces=false,
    breaklines=true,
    frame=lines,
    backgroundcolor=\color{background},
    literate=
     *{0}{{{\color{numb}0}}}{1}
      {1}{{{\color{numb}1}}}{1}
      {2}{{{\color{numb}2}}}{1}
      {3}{{{\color{numb}3}}}{1}
      {4}{{{\color{numb}4}}}{1}
      {5}{{{\color{numb}5}}}{1}
      {6}{{{\color{numb}6}}}{1}
      {7}{{{\color{numb}7}}}{1}
      {8}{{{\color{numb}8}}}{1}
      {9}{{{\color{numb}9}}}{1}
      {:}{{{\color{punct}{:}}}}{1}
      {,}{{{\color{punct}{,}}}}{1}
      {\{}{{{\color{delim}{\{}}}}{1}
      {\}}{{{\color{delim}{\}}}}}{1}
      {[}{{{\color{delim}{[}}}}{1}
      {]}{{{\color{delim}{]}}}}{1},
}

\definecolor{lightgray}{rgb}{.9,.9,.9}
\definecolor{darkgray}{rgb}{.4,.4,.4}
\definecolor{purple}{rgb}{0.65, 0.12, 0.82}
\lstdefinelanguage{JavaScript}{
  keywords={break, case, catch, continue, debugger, default, delete, do, else, false, finally, for, function, if, in, instanceof, new, null, return, switch, this, throw, true, try, typeof, var, void, while, with},
  morecomment=[l]{//},
  morecomment=[s]{/*}{*/},
  morestring=[b]',
  morestring=[b]",
  ndkeywords={class, export, boolean, throw, implements, import, this},
  keywordstyle=\color{blue}\bfseries,
  ndkeywordstyle=\color{darkgray}\bfseries,
  identifierstyle=\color{black},
  commentstyle=\color{numb}\ttfamily,
  stringstyle=\color{punct}\ttfamily,
  sensitive=true,
  stepnumber=1,
  backgroundcolor=\color{background},
  frame=lines,
  literate=
     *{0}{{{\color{numb}0}}}{1}
      {1}{{{\color{numb}1}}}{1}
      {2}{{{\color{numb}2}}}{1}
      {3}{{{\color{numb}3}}}{1}
      {4}{{{\color{numb}4}}}{1}
      {5}{{{\color{numb}5}}}{1}
      {6}{{{\color{numb}6}}}{1}
      {7}{{{\color{numb}7}}}{1}
      {8}{{{\color{numb}8}}}{1}
      {9}{{{\color{numb}9}}}{1}
      {:}{{{\color{punct}{:}}}}{1}
      {,}{{{\color{punct}{,}}}}{1}
      {\{}{{{\color{delim}{\{}}}}{1}
      {\}}{{{\color{delim}{\}}}}}{1}
      {[}{{{\color{delim}{[}}}}{1}
      {]}{{{\color{delim}{]}}}}{1}
      {\ \ }{{\ }}1,
}

\lstset{
   language=JavaScript,
   extendedchars=true,
   basicstyle=\footnotesize\ttfamily,
   showstringspaces=false,
   showspaces=false,
   numbers=left,
   numberstyle=\footnotesize,
   numbersep=9pt,
   tabsize=2,
   breaklines=true,
   showtabs=false,
   captionpos=b
}

% Custom
\usepackage{subfiles}
\usepackage[bottom]{footmisc}
\newcommand{\mytilde}{\raise.17ex\hbox{$\scriptstyle\mathtt{\sim}$}}

% The aalto-thesis package provides typesetting instructions for the
% standard master's thesis parts (abstracts, front page, and so on)
% Load this package second-to-last, just before the hyperref package.
% Options that you can use:
%   mydraft - renders the thesis in draft mode.
%             Do not use for the final version.
%   doublenumbering - [optional] number the first pages of the thesis
%                     with roman numerals (i, ii, iii, ...); and start
%                     arabic numbering (1, 2, 3, ...) only on the
%                     first page of the first chapter
%   twoinstructors  - changes the title of instructors to plural form
%   twosupervisors  - changes the title of supervisors to plural form
\usepackage[doublenumbering,twoinstructors]{aalto-thesis}

\hyphenation{lu-mi-no-pho-res}

\RequirePackage[pdftex]{hyperref}
\hypersetup{colorlinks=false,raiselinks=false,breaklinks=true}
\hypersetup{pdfborder={0 0 0}}
\hypersetup{bookmarksnumbered=true}
\hypersetup{bookmarksopen=true,bookmarksopenlevel=1}

% ------------------------------------------------------------------
\newcommand{\TITLE}{LuminoTrace: photoluminescence based product authentication for smartphones}
\newcommand{\FTITLE}{LuminoTrace: fotoluminesenssi-perusteinen aitouden tunnistaminen älypuhelimissa}

\newcommand{\SUBTITLE}{}
\newcommand{\FSUBTITLE}{}

\newcommand{\DATE}{\today}
\newcommand{\FDATE}{\today}


% Supervisors and instructors
% ------------------------------------------------------------------
\newcommand{\SUPERVISOR}{Professor Petri Vuorimaa, Aalto SCI}
\newcommand{\FSUPERVISOR}{Professori Petri Vuorimaa, Aalto SCI}

\newcommand{\INSTRUCTOR}{Jari Kleimola D.Sc. (Tech.)\\
Jouni Paltakari Professor (Tech.), Aalto CHEM}
\newcommand{\FINSTRUCTOR}{Jari Kleimola TkT\\
Jouni Paltakari Professori (tek.), Aalto CHEM}

% Other stuff
% ------------------------------------------------------------------
\newcommand{\PROFESSORSHIP}{Media Technology}
\newcommand{\FPROFESSORSHIP}{Viestintätekniikka}

% Professorship code is the same in all languages
\newcommand{\PROFCODE}{AS-75}

\newcommand{\KEYWORDS}{product authentication, photoluminescence, smartphone, mobile application, hybrid, counterfeit}
\newcommand{\FKEYWORDS}{aitouden tunnistaminen, todentaminen, fotoluminesenssi, älypuhelin, mobiiliapplikaatio, hybridi, väärennös}

\newcommand{\LANGUAGE}{English}
\newcommand{\FLANGUAGE}{Englanti}

% Author is the same for all languages
\newcommand{\AUTHOR}{Marko Raatikka}

% Currently the English versions are used for the PDF file metadata
% Set the PDF title
\hypersetup{pdftitle={\TITLE\ \SUBTITLE}}
% Set the PDF author
\hypersetup{pdfauthor={\AUTHOR}}
% Set the PDF keywords
\hypersetup{pdfkeywords={\KEYWORDS}}
% Set the PDF subject
\hypersetup{pdfsubject={Master's Thesis}}

% \linespread{1} % This is the default
\linespread{1}

% Bibliography style
\bibliographystyle{acm}
% Plainnat is a plain style that works with both numbered and name citations.
% \bibliographystyle{plainnat}


% Extra hyphenation settings
% ------------------------------------------------------------------
% You can list here all the files that are not hyphenated correctly.
% You can provide many \hyphenation commands and/or separate each word
% with a space inside a single command. Put hyphens in the places where
% a word can be hyphenated.
% Note that (by default) LaTeX will not hyphenate words that already
% have a hyphen in them (for example, if you write ``structure-modification
% operation'', the word structure-modification will never be hyphenated).
% You need a special package to hyphenate those words.
\hyphenation{di-gi-taa-li-sta yksi-suun-tai-sta}



% The preamble ends here, and the document begins.
% Place all formatting commands and such before this line.
% ------------------------------------------------------------------
\begin{document}
% This command adds a PDF bookmark to the cover page. You may leave
% it out if you don't like it...
\pdfbookmark[0]{Cover page}{bookmark.0.cover}
% This command is defined in aalto-thesis.sty. It controls the page
% numbering based on whether the doublenumbering option is specified
\startcoverpage

% Cover page
% ------------------------------------------------------------------
% Options: finnish, english, and swedish
% These control in which language the cover-page information is shown
\coverpage{english}


% Abstracts
% ------------------------------------------------------------------
% Include an abstract in the language that the thesis is written in,
% and if your native language is Finnish or Swedish, one in that language.

% Abstract in English
% ------------------------------------------------------------------
\thesisabstract{english}{Estimations by ICC in 2008 indicated that the value of internationally traded counterfeit products would reach upto 1770 billion US dollars by 2015 \cite{icc}. At the same time high-volume commodities are becoming an increasily more popular target. While brand owners might be willing to tradeoff low-cost and ease of use for greater security to protect their high-end products, for high-volume commodities the cost-effectiveness and simplicity of the product authentication solution are often imperative.

This thesis introduces \emph{LuminoTrace} -- a photoluminescence-based product authentication technology for smartphones -- that seeks to combine a covert, low-cost and portable feature set for high-security product authentication needs. This is achieved by smartphone-based colorimetrical and temporal analysis of luminophores to construct a \emph{fingerprint} that can be used to uniquely authenticate a product. To facilitate the requirements of end-to-end product authencation needs a proof-of-concept cloud architecture was implemented.

The results of this thesis indicate that the technology is not yet ready for mainstream adoption. Outstanding challenges relate to both the current smartphone technology and the luminophores. For efficient temporal analysis natively supported RAW and timed burst capture capabilities are required. Furthermore, artifical fabrication is called for to create cost-effective luminophores with the desired properties (photobleach resistance, narrow absorption wavelength, short lifetime).

In the future \emph{tunable fluorescent materials} may provide the means for cost-effective luminophore fabrication. Emerging mobile technologies such as modular smartphones like \emph{Google Ara} may enable seamless integration of the required light source with the smartphone to facilitate a wider adoption of photoluminescence-based product authentication solutions. The technique developed in this thesis however remains applicable for small-scale, internal authentication needs.}

\thesisabstract{finnish}{Kansainvälisen kauppakamarin (ICC) laskelmien mukaan kansainvälisesti kaupattujen väärennettyjen tuotteiden arvo kasvanee 1770 miljardiin dollariin vuonna 2015 \cite{icc}. Suuren volyymin tuotteet ovat jatkuvasti suositumpi väärennyskohde. Vaikka tuotemerkin omistajat olisivat valmiita hyväksymään korkeammat kustannukset ja huonomman siirrettävyyden (engl. \emph{portability}) paremman turvallisuustason nimissä, suuren volyymin hyödykkeille turvallisuusratkaisun kustannustehokkuus ja helppokäyttöisyys ovat usein tärkeitä.

Tämän diplomityön tulos on \emph{LuminoTrace} -- fotoluminesenssiin perustuva aitouden tunnistus teknologia älypuhelimille -- joka pyrkii yhdistämään peitetyn, edullisen ja helposti siirrettävän turvallisuusratkaisun ominaisuudet korkeisiin turvallisuustarpeisiin. Teknologia perustuu älypuhelimella suoritettavaan luminoforin kolorimetriseen ja aikaerotteiseeen analyysiin tuotteen autentikoimiseen soveltuvan uniikin \emph{sormenjäljen} luomiseksi. Menetelmän tueksi toteutettiin pilviarkkitehtuuri kokonaisvaltaisen tuotteen todentamisratkaisun demonstroimiseksi.

Työn tulokset osoittavat, ettei kyseinen ratkaisu ole vielä sellaisenaan sovellettavissa laajempaan käyttöön. Jäljellä olevat haasteet liittyvät sekä nykyisen älypuhelinteknologian että luminoforien ominaisuuksiin. Tarkan aikaerotteisen analyysin suorittamiseksi älypuhelimen tulisi tukea raakakuvaformaattia (RAW) sekä ajastettua sarjakuvausta. Myös valmistusmenetelmiä sopivien luminoforien (kapea viritysaallonpituusalue, lyhyt emissio) valmistamiseksi vaaditaan.

Tulevaisuudessa \emph{muokattavat fluoresoivat materiaalit} saattavat auttaa kustannustehokkaiden luminoforien valmistuksessa. Orastavat älypuhelinteknologiat kuten modulaarinen älypuhelinarkkitehtuuri (Google Ara) saattavat mahdollistaa vaaditun valolähteen saumattoman integraation, ja siten myös fotoluminesenssi-perusteisen aitouden tunnistusratkaisujen laajemman käyttöönoton. Tässä työssä kehitetty ratkaisu soveltunee kuitenkin sisäisiin, pienen skaalan käyttötarpeisiin.}

\selectlanguage{english}

\addcontentsline{toc}{chapter}{Acknowledgments}

\chapter*{Acknowledgments}
First and foremost I would like thank my instructor Dr. Jari Kleimola for his professionalism, positive outlook and utmost patience during the guidance of my work. Your advice was invaluable, a great example of which was the idea for the \emph{cross-modal} communication with the camera module.

Many thanks to MSc. Päivi Kuosmanen (Aalto CHEM) for preparing the taggants for the experiment. I would also like to extend my gratitude to Prof. Jouni Paltakari and my supervisor Prof. Petri Vuorimaa for acting as the facilitators for such an interesting project. Thank you for letting me take the time it took to finish the thesis.

Last, I should not forget friends and family, whose encouraging words and countless ways of reminding me of the work ahead helped push this over the finish line. Special kudos to my sister Minna, mom Mirja and dear Riina for their invaluable support.

\vskip 10mm

\noindent Espoo, \DATE
\vskip 5mm
\noindent\AUTHOR

\cleardoublepage

% Acronyms
% ------------------------------------------------------------------

\subfile{abbreviations.tex}

% Table of contents
% ------------------------------------------------------------------
\cleardoublepage
% This command adds a PDF bookmark that links to the contents.
% You can use \addcontentsline{} as well, but that also adds contents
% entry to the table of contents, which is kind of redundant.
% The text ``Contents'' is shown in the PDF bookmark.
\pdfbookmark[0]{Contents}{bookmark.0.contents}

\begingroup
\makeatletter
% Redefine the \chapter* header macro to remove vertical space
\def\@makeschapterhead#1{%
  %\vspace*{50\p@}% Remove the vertical space
  {\parindent \z@ \raggedright
    \normalfont
    \interlinepenalty\@M
    \Huge \bfseries  #1\par\nobreak
    \vskip 40\p@
  }}
\makeatother

\tableofcontents
\endgroup

% List of tables
% ------------------------------------------------------------------
% You only need a list of tables for your thesis if you have very
% many tables. If you do, uncomment the following two lines.
% \cleardoublepage
% \listoftables

% Table of figures
% ------------------------------------------------------------------
% You only need a list of figures for your thesis if you have very
% many figures. If you do, uncomment the following two lines.
\cleardoublepage
\listoffigures

% The following label is used for counting the prelude pages
\label{pages-prelude}

\cleardoublepage

%%%%%%%%%%%%%%%%% The main content starts here %%%%%%%%%%%%%%%%%%%%%
% ------------------------------------------------------------------
% This command is defined in aalto-thesis.sty. It controls the page
% numbering based on whether the doublenumbering option is specified
\startfirstchapter

% Add headings to pages (the chapter title is shown)
\pagestyle{headings}

% The contents of the thesis are separated to their own files.
% Edit the content in these files, rename them as necessary.
% ------------------------------------------------------------------
\subfile{chapters/1-intro.tex}

\subfile{chapters/2-background.tex}

\subfile{chapters/3-design-impl.tex}

\subfile{chapters/4-results.tex}

\subfile{chapters/5-discussion.tex}

\subfile{chapters/6-conclusions.tex}

% Load the bibliographic references
% ------------------------------------------------------------------
\bibliography{references}


% Appendices go here
% ------------------------------------------------------------------
% If you do not have appendices, comment out the following lines
\appendix
\subfile{appendices.tex}

% End of document!
% ------------------------------------------------------------------
% The LastPage package automatically places a label on the last page.
% That works better than placing a label here manually, because the
% label might not go to the actual last page, if LaTeX needs to place
% floats (that is, figures, tables, and such) to the end of the
% document.
\end{document}
